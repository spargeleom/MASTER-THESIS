\chapter{システムの線形性とその同定}
前章で導出した物理モデルにはバルクや粘性項などのパラメータが含まれている.
これらのパラメータを全て知ることができればシリンダのモデルを得ることが可能であるが,現実的に組み合わせることは難しい.
そこで本章ではシステム同定を用いて油圧システムのモデルの構築を行う.
バルブへの電圧入力からシリンダ先端の位置及び先端で発生する力までのモデルを,システム同定を用いて導出する.
力の同定ではシリンダ先端を固定した状態で同定をする.
位置の同定にあたっては無負荷状態で行う.
%同定にあたってはシステムの周波数応答を調べて特徴を把握したのちに最小自乗法によるモデルの同定,およびM系列を用いた同定を行いそれぞれを比較する.
\section{実験機とその構成}
本研究で使用する油圧システムの実験機を紹介する.
\section{先端で発生する力までの線形性とモデルの同定}
\subsection{線形性調査}
\subsection{周波数応答}
\subsection{最小自乗法による伝達関数モデルの同定}
\section{位置までの同定}
\subsection{線形性調査}
\subsection{周波数応答}
\section{M系列による同定}
\subsection{M系列の性質}
\subsection{システム同定(入力から位置)}
\subsection{システム同定(入力から力)}

