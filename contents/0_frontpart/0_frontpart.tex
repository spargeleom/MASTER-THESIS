% タイトルページ
\begin{titlepage}
\centering
\vspace*{40truept}
{\Large 平成30年度 修士論文} \\ % 年度
\vspace{40truept} 
 {\Large 題目} \\
 \vspace{10truept} 
{\LARGE \textbf{タイトル}}\\ % タイトル
\vspace{10truept}
{\Large --- サブタイトル ---}\\ % サブタイトル. なければコメントアウト
\vspace{120truept}
{\Large 指導教員}\\ %
 \vspace{10truept} 
{\Large 石川 将人 教授}\\ % 指導教員
\vspace{60truept}
{\Large 大阪大学大学院 工学研究科 機械工学専攻}\\ % 学科
 \vspace{10truept} 
{\Large 学籍番号 28E17076}\\ % 学籍番号
\vspace{20truept}
{\LARGE 名前}\\ % 著者
\vspace{80truept}
{\Large 2016年2月xx日} % 提出日
\end{titlepage}
\cleardoublepage
% アブストラクト
\chapter*{\huge 概要}
\vskip2\Cvs

卒業論文を\LaTeX で書くときに参考になればと思い作りました. なぜかコンパイルできない, Wordみたいな微調整ができなくて体裁が整わないなどの``\LaTeX あるある''で, 無駄に時間を費やさないように, 本来時間を割くべきところにきちんと時間を割けるようにしましょう. 

本テンプレートは使用を強要するものではありません. すでにShareフォルダ内に, 末岡先生が作られた大須賀研用のテンプレがありますのでそれを用いてもらっても構いません. あるいは自分で論文体裁を整えてもらっても構いません. 要するに論文が書ければそれでいいのです. 





\section*{\huge Abstract}
\vskip2\Cvs
This paper discusses ...
%
%


\newpage

%目次
\tableofcontents   %目次
\thispagestyle{plain}
%\newpage
\listoffigures %図目次 図が少なければいらないかも
%\newpage
\listoftables %表目次 表が少なければいらないかも
%\newpage

