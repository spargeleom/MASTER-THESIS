% タイトルページ
\begin{titlepage}
\centering
\vspace*{40truept}
{\Large 平成30年度 修士論文} \\ % 年度
\vspace{40truept} 
 {\Large 題目} \\
 \vspace{10truept} 
{\LARGE \textbf{圧力センサを用いた油圧アクチュエータの力制御}}\\ % タイトル
\vspace{10truept}
%{\Large --- サブタイトル ---}\\ % サブタイトル. なければコメントアウト
\vspace{120truept}
{\Large 指導教員}\\ %
 \vspace{10truept} 
{\Large 石川 将人 教授\\南 裕樹 講師}\\ % 指導教員
\vspace{60truept}
{\Large 大阪大学大学院 工学研究科 機械工学専攻}\\ % 学科
 \vspace{10truept} 
{\Large 学籍番号 28E17076}\\ % 学籍番号
\vspace{20truept}
{\LARGE 吉田 侑史}\\ % 著者
\vspace{80truept}
{\Large 2019年2月7日} % 提出日
\end{titlepage}
\cleardoublepage
% アブストラクト
\chapter*{\huge 概要}
%\vskip2\Cvs
油圧アクチュエータは耐衝撃性や出力質量比が大きいという特徴があり,福島第一原子力発電所の燃料デブリ取り出しに向けて開発されている大型油圧マニピュレータPA-2000に用いられている.
廃炉を進めるために,手先の位置および力を制御することが求められており,本論文では力制御について取り組む.
油圧マニピュレータの手先における力制御を行うにあたり,圧力センサから力を推定するで力センサが不要となるため,省線化やマニピュレータ先端に取り付けるツール設計の簡略化をすることができる.

本論文では力制御実現に向けた第一段階として,油圧シリンダにおける力推定手法の構築および制御を行う.
はじめに,油圧シリンダにシステム同定を適用し,力の推定手法を構築する.
つぎに,推定した力を用いて力制御を行い,PID制御や$H_\infty$制御を適用し,$H_\infty$制御において外乱抑制ができることを確認した.
そして,位置制御と力制御を統合させたPosition-Force Integrated Control(PFIC)を提案し,コンプライアンス制御を行った.
そしてコンプライアンス制御においてフィードバック変調器を適用し,定常特性が改善されることを示した.
% 油圧アクチュエータは耐衝撃性や出力質量比が大きいという特徴があり,ロボティクスの分野において着目されている.
%
%


\newpage

%目次
\tableofcontents   %目次
\thispagestyle{plain}
%\newpage
\listoffigures %図目次 図が少なければいらないかも
%\newpage
\listoftables %表目次 表が少なければいらないかも
%\newpage

