\chapter{力制御}
\label{sec:ForceControl}
油圧システムの力制御については,センサを取り付けて直接力を測定したり,\eqnname\eqref{eq:fthr}による推力をそのままシリンダの出力として用いる方法がとられてきた\cite{semini2010design,semini2010hyq,川端健太郎20141a1,岡田大貴2017多自由度油圧駆動ロボットのシリンダ圧に基づく手先負荷力推定による力覚フィードバック}.
本章では,$\GfthrTofmsr$を用いて出力を推定する推定アルゴリズムおよび制御器の設計とその比較を行う.

\section{力推定アルゴリズム}
力制御をおこなうためのシステムのブロック線図を\figname\ref{fig4:ForceEstimateControl}に示す.
油圧シリンダのhead側およびrod側の圧力と受圧面積から\eqnname\eqref{eq:fthr}を用いて推力$\fthr$を計算して伝達関数$\GfthrTofmsr$へ入力し,その出力を推定出力$\fest$としてフィードバックする.
実際に発生している実測出力$\fmsr$はLoad Cellにより測定される.

\begin{figure}[t]
    \centering
        \includegraphics[keepaspectratio, scale=1.0]{contents/ForceControl/figure/TikZ/ForceEstimateControl-crop.pdf}
%        \input{contents/ForceControl/figure/TikZ/ForceEstimateControl.tex}
        \caption{Block Diagram for Force Control}
        \label{fig4:ForceEstimateControl}
\end{figure}


\section{PID制御とI-PD制御}
PID制御器(\figname\ref{fig4:PID})とI-PD制御器(\figname\ref{fig4:IPD})による応答を調べる.
I-PD制御器を利用するのは,目標値の急峻な変化に対するオーバーシュートを抑制させることを目的とするためである.
$s/(\tau s+1)$は近似微分器であり,$\tau=0.005$とした.
制御器におけるそれぞれのゲイン$K_P$,$K_I$,$K_D$は$\GinTofmsr$に対し限界感度法を適用して決定し,$K_P=8.4,~K_I =168,~K_D=0.1$とした.
\begin{figure}[t]
    \begin{minipage}{\minipageratio\hsize}
    \centering
        \includegraphics[keepaspectratio, scale = \minifigscale]{contents/ForceControl/figure/PIDcontroller-crop.pdf}
%        \begin{tikzpicture}
    [
        scale =1,
    % 線と文字の間の間隔を設定
    every node/.style={outer sep=0.12cm, inner sep=0},
    % 矢印の設定
    arrow/.style={-{Stealth[length=0.25cm]}, thick},
    % ブロック
    block/.style={rectangle, draw, minimum height = 1cm,
    minimum width=0.8cm, thick, outer sep = 0},
    block2/.style={rectangle, draw, minimum height = 2.5cm,
    minimum width=1.2cm, thick, outer sep = 0},
    % 加え合わせ点
    sum/.style={thick, circle, draw, inner sep=0,
    minimum size=6pt, outer sep=0},
    % 引き出し点
    point/.style={radius=2pt},
    %%
    gain/.style = {regular polygon, regular polygon sides=3,
        draw, fill=white, minimum width=0.8cm,
        inner sep=0, outer sep=0mm,
        shape border rotate=-90}
    ]
    \node [gain](KP){$K_\mathrm{P}$};
    \node [gain, below=0.7cm of KP] (KI){$K_\mathrm{I}$};
    \node [gain, below=0.7cm of KI](KD){$K_\mathrm{D}$};    
    \node [block,left=0.5cm of KD] (dev){~$\dfrac{s}{\tau s + 1}$~~};
    \node [block,left=0.8cm of KI] (int){$\dfrac{1}{s}$};
    \node [sum, right=0.6 of KI](sum){};
    \node[sum,left=1.2cm of int](u){};
    \node[left=0.6cm of u](SP){Set Point};
    \node[below=2.5cm of u](y){Process Variable};
    
    %\node[left=2cm of int](u){};



   \fill[point](int)+(-1.1,0) circle coordinate(u1);
   \draw[arrow](u1)|-(KP);
   \draw[arrow](KP)-|(sum)node[left=-0.05cm,yshift = 0.45cm]{$+$};
   \draw[arrow](u)--(int)node[above,pos=0.1]{};
   \draw[arrow](int)--(KI);
   \draw[arrow](KI)--(sum)node[left=0.2cm,yshift = -0.2cm]{$+$};
   \draw[arrow](u1)|-(dev);
   \draw[arrow](dev)--(KD);
   \draw[arrow](KD)-|(sum)node[right=-0.05cm,yshift = -0.4cm]{$+$};
   \draw[arrow](sum)-- +(1,0)node[right]{Input};
   \draw[arrow](SP) -- (u)node[left=0.2cm,yshift = -0.2cm]{$+$};
   \draw[arrow](y)--(u)node[left=-0.05cm,yshift = -0.4cm]{$-$};
%    \node [block] (K){~Controller~~};
%    \node [block2, right=0.6 of K] (G){\begin{tabular}{l}Hydraulic\\Cylinder\end{tabular}};
%    \node [sum, left=0.6of K] (sum){};    
%    \coordinate (G1) at ($(G.north east)!0.25!(G.south east)$);
%    \coordinate (G2) at ($(G.north east)!0.75!(G.south east)$);
%    \node [gain, right=0.6of G1] (Ah){$\Ah$};
%    \node [gain, right=0.6of G2] (Ar){$\Ar$};
%    \node [sum,right=1.5 of G](sum2){};
%    \node [block,right=-1cm of sum2,yshift = -2cm] (Gin2f){$\GinTofmsr$};
%    
%    
%    \draw[arrow] (sum) -- (K);
%    \draw[arrow] (K) -- (G) node [above, pos=0.5] {};
%    \draw[arrow] (G1) -- (Ah) node[above,pos = 0.5]{$\phs$};
%    \draw[arrow] (G2) -- (Ar) node[above,pos = 0.5]{$\prs$};
%    \draw[arrow] (Ah) -| (sum2)node[left, yshift=+10pt] {$+$};
%    \draw[arrow] (Ar) -| (sum2)node[right=0.3cm,yshift=0.3cm] {$\fthr$}
%    node[left, yshift=-10pt] {$-$};
%    \draw[arrow] (sum2) -|+(0.5,-1.3)-- +(-1.3,-1.3) |- (Gin2f);
%    \draw[arrow] (Gin2f) -- +(2,0)node[right] {\begin{tabular}{l}Estimated\\Force\end{tabular}};
%    \fill [point] (Gin2f)+(1.5, 0) circle coordinate (y);
%    
%    \draw[arrow] (G.north) |- +(2,1) node[right]{Measured Force};
%    \draw[arrow] (sum.west)+(-0.5, 0) node[left]{\begin{tabular}{l}Set\\Point\end{tabular}} -- (sum.west)
%    node[below, xshift=-10pt]{$+$};
%    \draw [arrow] (y) -- +(0, -1) -| (sum) node[left, yshift=-15pt] {$-$};

\end{tikzpicture}
        \subcaption{PID Controller}
        \label{fig4:PID}
    \end{minipage}
    \begin{minipage}{\minipageratio\hsize}
    \centering
        \includegraphics[keepaspectratio, scale = \minifigscale]{contents/ForceControl/figure/I-PDcontroller-crop.pdf}
%        \begin{tikzpicture}
    [
        scale =1,
    % 線と文字の間の間隔を設定
    every node/.style={outer sep=0.12cm, inner sep=0},
    % 矢印の設定
    arrow/.style={-{Stealth[length=0.25cm]}, thick},
    % ブロック
    block/.style={rectangle, draw, minimum height = 1cm,
    minimum width=0.8cm, thick, outer sep = 0},
    block2/.style={rectangle, draw, minimum height = 2.5cm,
    minimum width=1.2cm, thick, outer sep = 0},
    % 加え合わせ点
    sum/.style={thick, circle, draw, inner sep=0,
    minimum size=6pt, outer sep=0},
    % 引き出し点
    point/.style={radius=2pt},
    %%
    gain/.style = {regular polygon, regular polygon sides=3,
        draw, fill=white, minimum width=0.8cm,
        inner sep=0, outer sep=0mm,
        shape border rotate=-90}
    ]
    \node [sum](sum1){};
    \node [gain, right = 2.5cm of sum1] (KI){$K_\mathrm{I}$};
    \node [block,left=0.8cm of KI] (int){$\dfrac{1}{s}$};

\node[below=4cm of sum1](y){Process Variable};
    \node[left=1cm of sum1](r){Set Point};
       
    \coordinate (y1) at ($(sum1)!0.4!(y)$);
    \coordinate (y2) at ($(sum1)!0.8!(y)$);
    \node [gain, right=2cm of y1] (KP){$K_\mathrm{P}$};
    \node [gain, right=2.5cm of y2](KD){$K_\mathrm{D}$};    
    \node [block,left=0.5cm of KD] (dev){~$\dfrac{s}{\tau s + 1}$~~};

    \node [sum,right=0.5cm of KP](sum2){};
    \node[right=0.8cm of sum2](input){Input};
    
    \fill[point](y1) circle coordinate;
    \fill[point](y2) circle coordinate;
    
    \draw[arrow](y)--(sum1)node[left=-0.05cm,yshift = -0.4cm]{$-$};
    \draw[arrow](r)--(sum1)node[left=0.2cm,yshift = -0.2cm]{$+$};
    \draw[arrow](sum1)--(int);
    \draw[arrow](int)--(KI);
    \draw[arrow](KI)-|(sum2)node[left=-0.05cm,yshift = 0.45cm]{$+$};
    \draw[arrow](y1)--(KP);
    \draw[arrow](KP)--(sum2)node[left=0.2cm,yshift = -0.2cm]{$-$};
    \draw[arrow](y2)--(dev);
    \draw[arrow](dev)--(KD);
    \draw[arrow](KD)-|(sum2)node[right=-0.05cm,yshift = -0.4cm]{$-$};
    \draw[arrow](sum2)--(input);
    
    
    
%    \fill[point](int)+(-1.2,0) circle coordinate(u1);
%    \draw[arrow](u1)|-(KP);
%    \draw[arrow](KP)-|(sum)node[left=-0.05cm,yshift = 0.45cm]{$+$};
%    \draw[arrow](u)--(int)node[above,pos=0.1]{error};
%    \draw[arrow](int)--(KI);
%    \draw[arrow](KI)--(sum)node[left=0.2cm,yshift = -0.2cm]{$+$};
%    \draw[arrow](u1)|-(dev);
%    \draw[arrow](dev)--(KD);
%    \draw[arrow](KD)-|(sum)node[right=-0.05cm,yshift = -0.4cm]{$+$};
%    \draw[arrow](sum)-- +(1,0)node[above]{input};


%    \node [block] (K){~Controller~~};
%    \node [block2, right=0.6 of K] (G){\begin{tabular}{l}Hydraulic\\Cylinder\end{tabular}};
%    \node [sum, left=0.6of K] (sum){};    
%    \coordinate (G1) at ($(G.north east)!0.25!(G.south east)$);
%    \coordinate (G2) at ($(G.north east)!0.75!(G.south east)$);
%    \node [gain, right=0.6of G1] (Ah){$\Ah$};
%    \node [gain, right=0.6of G2] (Ar){$\Ar$};
%    \node [sum,right=1.5 of G](sum2){};
%    \node [block,right=-1cm of sum2,yshift = -2cm] (Gin2f){$\GinTofmsr$};
%    
%    
%    \draw[arrow] (sum) -- (K);
%    \draw[arrow] (K) -- (G) node [above, pos=0.5] {};
%    \draw[arrow] (G1) -- (Ah) node[above,pos = 0.5]{$\phs$};
%    \draw[arrow] (G2) -- (Ar) node[above,pos = 0.5]{$\prs$};
%    \draw[arrow] (Ah) -| (sum2)node[left, yshift=+10pt] {$+$};
%    \draw[arrow] (Ar) -| (sum2)node[right=0.3cm,yshift=0.3cm] {$\fthr$}
%    node[left, yshift=-10pt] {$-$};
%    \draw[arrow] (sum2) -|+(0.5,-1.3)-- +(-1.3,-1.3) |- (Gin2f);
%    \draw[arrow] (Gin2f) -- +(2,0)node[right] {\begin{tabular}{l}Estimated\\Force\end{tabular}};
%    \fill [point] (Gin2f)+(1.5, 0) circle coordinate (y);
%    
%    \draw[arrow] (G.north) |- +(2,1) node[right]{Measured Force};
%    \draw[arrow] (sum.west)+(-0.5, 0) node[left]{\begin{tabular}{l}Set\\Point\end{tabular}} -- (sum.west)
%    node[below, xshift=-10pt]{$+$};
%    \draw [arrow] (y) -- +(0, -1) -| (sum) node[left, yshift=-15pt] {$-$};

\end{tikzpicture}
        \subcaption{I-PD Controller}
        \label{fig4:IPD}
    \end{minipage}
    \caption{PID and I-PD Controller}
\end{figure}

それぞれの制御器におけるステップ応答を\figname\ref{fig4:SRofPIDandIPD}に示す.
\figname\ref{fig4:SR_fest}が推定値の応答,\figname\ref{fig4:SR_fmsr}が実測出力の応答である.
\figname\ref{fig4:SR_fest}より推定値$\fest$は目標値に対し定常偏差なく追従している.
実測出力$\fmsr$の応答は,\figname\ref{fig4:SR_fmsr}より定常偏差が残っていることが確認されるが,偏差は5\,\%以内に収まっており,推定値をフィードバックすることにより実測出力を制御することができているといえる.
また,I-PD制御器におけるオーバーシュートがPID制御器に比べて抑制されており,I-PD制御器を導入した意図を満たしている.

\begin{figure}[tbp]
    \begin{minipage}{\minipageratio\hsize}
        \centering
        \includegraphics[keepaspectratio, scale = \minifigscale]{contents/ForceControl/figure/1115_PIDandI-PDestforce_step.pdf}
        \subcaption{Estimated Force:$\fest$}
        \label{fig4:SR_fest}
    \end{minipage}
    \begin{minipage}{\minipageratio\hsize}
        \centering
        \includegraphics[keepaspectratio, scale = \minifigscale]{contents/ForceControl/figure/1115_PIDandI-PDforce_step.pdf}
        \subcaption{Measured Force:$\fmsr$}
        \label{fig4:SR_fmsr}
    \end{minipage}
    \caption{Sine Response of Estimated Force and Measured Force (PID Controller and I-PD Controller)}
    \label{fig4:SRofPIDandIPD}
\end{figure}


\section{$H_\infty$制御}
対象のシステムに対して,$H_\infty$制御の適用をする.
$H_\infty$制御の適用により,モデル化誤差の吸収および外乱抑制が期待される\cite{三平満司1997実用的な,平田201703}.
力制御を行う際には環境との接触を伴うため,環境から振動など様々な外乱を受けることになる.
その点においても,外乱抑制を行う制御の適用は意味あるものとなる.
\subsection{対象システムの状態空間表現}
$H_\infty$制御器を設計するにあたり,対象システムのモデルを状態空間で表現する必要がある.
なお,$H_\infty$制御器の設計には\ref{sec:LD伝達関数モデル(力)}節で求めた最小自乗法による伝達関数モデル,$\GinTofmsr$を用いる.
$\GinTofmsr$を状態空間表現に書き直すと,\eqnname\eqref{eq:ss_in2fmsr}となる.
導出の詳細は付録\ref{sec:tf2ss}に示す.
\begin{align}
    \label{eq:ss_in2fmsr}
    \begin{split}
        &\begin{cases}
            \dot{x_\mathrm{p}}
            &=A_\mathrm{p}x_\mathrm{p}
            +B_\mathrm{p}
            u_\mathrm{p} \\[10pt]
            y_\mathrm{p} &=
            C_\mathrm{p}x_\mathrm{p}
        \end{cases}\\[10pt]
        &x_\mathrm{p} = 
        \begin{bmatrix}
            x_1(t-0.006)\\
            x_2(t)
        \end{bmatrix}\\
        &u_\mathrm{p} = u(t-0.016)\\
        &A_\mathrm{p} = 
        \begin{bmatrix}
            -0.34 & 0\\
            1 & 0
        \end{bmatrix}\\
        &B_\mathrm{p} = 		
        \begin{bmatrix}
            3.4\\
            0
        \end{bmatrix}\\
        &C_\mathrm{p} = 
        \begin{bmatrix}	
            0 & 123
        \end{bmatrix}
    \end{split}
\end{align}

\clearpage
\subsection{$H_\infty$制御器の設計}
\label{sec:H無限大制御器の設計}
$H_\infty$制御器の設計にあたってはむだ時間をそのまま扱うことはできないため,(ⅰ)むだ時間を無視した場合のコントローラ$K_{H_\infty}\mathrm{servo}$と(ⅱ)むだ時間を1次でPad\'e近似した場合のサーボ系コントローラ$K_{H_\infty\mathrm{servo}}^{\mathrm{Pad\acute{e}}}$,それぞれの設計を行う.
制御器を設計するシステムは\figname\ref{fig4:konngoukanndo}で示す出力端混合感度問題であり\footnote{SISOシステムの場合入力端混合感度問題と出力端混合感度問題は同一であり,本研究で扱っている対象のシステムはSISOシステムである.},破線で囲まれた部分を一般化プラントとして取扱う.
\figname\ref{fig4:konngoukanndo}中の各パラメータは\eqnname\eqref{eq:mixed_param}とし,コントローラを設計する.
設計手法の詳細は付録\ref{sec:Hinfty_sekkei}に示す.
\begin{align}
    \label{eq:mixed_param}
		W_T &= \frac{15(s+10)}{s+400}\\
		W_S &= \frac{5}{s+0.1}\\
		\varepsilon &= 0.001
\end{align}
\begin{figure}[t]
    \centering
        \includegraphics[keepaspectratio, scale=1.0]{contents/ForceControl/figure/konngoukanndo.pdf}
        \caption{Mixed $H_\infty$ Synthesis}
        \label{fig4:konngoukanndo}
\end{figure}

\subsection{サーボ系$H_\infty$制御器}
\label{sec:サーボ系H無限大制御器}
\ref{sec:H無限大制御器の設計}節で設計した$H_\infty$制御器をサーボ系に適用するため,コントローラに積分器を導入する.
一般に使用される$1/s$の形の積分器はコントローラから見ると不可観測かつ不安定になる\cite{三平満司1997実用的な}.
そこで,$(s+\alpha)/s$の形の積分器を,\figname\ref{fig4:konngoukanndo}を\figname\ref{fig4:servo_hinf}のように書き直して導入することで,サーボ系を構築することができる.
$\alpha$の値は正の実数であれば任意である.
$\alpha$の値を0から6まで2ずつ変化させて,ステップ応答を比較した結果を\figname\ref{fig4:crop-1115_JFPSHinf_step}および\figname\ref{fig4:crop-1115_JFPSHinfpade1_step}に示す.
\figname\ref{fig4:crop-1115_JFPSHinf_step}がむだ時間を無視して設計したサーボ系コントローラ$K_{H_\infty\mathrm{servo}}$を用いたときの応答,\figname\ref{fig4:crop-1115_JFPSHinfpade1_step}がむだ時間をPad\'e近似して設計したサーボ系コントローラ$K_{H_\infty\mathrm{servo}}^{\mathrm{Pad\acute{e}}}$を用いたときの応答を示す.
また,それぞれ,(a)が推定値の応答,(b)が実測出力の応答である.
\figname\ref{fig4:crop-1115_JFPSHinf_estforce_step}および\figname\ref{fig4:crop-1115_JFPSHinf_estforcepade1_step}において目標値に定常偏差なく追従していることから,設計したコントローラはどちらもサーボ系を構成できていることが確認できる.
$\alpha$が増大するごとに,収束時間が早く,オーバーシュートが大きくなっていることが確認でき,これらの兼ね合いから本研究では今後,$\alpha=4$としたサーボ系$H_\infty$制御器を用いる.
\begin{figure}[t]
    \centering
        \includegraphics[keepaspectratio, scale=1.0]{contents/ForceControl/figure/servo_hinf.png}
        \caption{Servo $H_\infty$ Controller}
        \label{fig4:servo_hinf}
\end{figure}

\begin{figure}[t]
    \begin{minipage}{\minipageratio\hsize}
    \centering
        \includegraphics[keepaspectratio, scale = \minifigscale]{contents/ForceControl/figure/1115/crop-1115_JFPSHinf_estforce_step.pdf}
        \subcaption{Estimated Force:$\fest$}
        \label{fig4:crop-1115_JFPSHinf_estforce_step}
    \end{minipage}
    \begin{minipage}{\minipageratio\hsize}
    \centering
        \includegraphics[keepaspectratio, scale = \minifigscale]{contents/ForceControl/figure/1115/crop-1115_JFPSHinf_force_step.pdf}
        \subcaption{Measured Force:$\fmsr$}
        \label{fig4:crop-1115_JFPSHinf_force_step}
    \end{minipage}
    \caption{Step Response of Estimated Force and Measured Force ($H_\infty$ servo controller w/o dead time)}   
    \label{fig4:crop-1115_JFPSHinf_step}
\end{figure}
\begin{figure}[t]
    \begin{minipage}{\minipageratio\hsize}
    \centering
        \includegraphics[keepaspectratio, scale = \minifigscale]{contents/ForceControl/figure/1115/crop-1115_JFPSHinf_estforcepade1_step.pdf}
        \subcaption{Estimated Force:$\fest$}
        \label{fig4:crop-1115_JFPSHinf_estforcepade1_step}
    \end{minipage}
    \begin{minipage}{\minipageratio\hsize}
    \centering
        \includegraphics[keepaspectratio, scale = \minifigscale]{contents/ForceControl/figure/1115/crop-1115_JFPSHinf_forcepade1_step.pdf}
        \subcaption{Measured Force:$\fmsr$}
        \label{fig4:crop-1115_JFPSHinf_forcepade1_step}
    \end{minipage}
    \caption{Step Response of Estimated Force and Measured Force ($H_\infty$ servo controller w/ Pad\'e approximation of dead time)}
    \label{fig4:crop-1115_JFPSHinfpade1_step}
\end{figure}

\clearpage

\section{I-PD制御と$H_\infty$制御の応答および外乱抑制効果の比較}
I-PD制御器と\ref{sec:サーボ系H無限大制御器}節で設計したサーボ系$H_\infty$制御器の応答,及び外乱が応答へ与える影響の比較を行う.
目標値として,ステップ目標および正弦波目標$f_\mathrm{target} = \sin(t)$を与える.
ステップ目標を与えたときの応答を\figname\ref{fig4:crop-1115_diff_step}に,正弦波目標を与えたときの応答を\figname\ref{fig4:crop-1115_diff_sin}に示す.
\figname\ref{fig4:crop-1115_diff_step}より,I-PD制御が最もオーバーシュートが少なく,整定時間も早いことが確認できる.
また,正弦波目標に対してはどの制御器を用いても目標値に追従しているが,実測出力$\fmsr$は正弦波目標に対して遅れていることが確認できる.
%%===========
%%FIGURE
%%===========
\begin{figure}[t]
    \begin{minipage}{\minipageratio\hsize}
    \centering
        \includegraphics[keepaspectratio, scale = \minifigscale]{contents/ForceControl/figure/1115/crop-1115_diff_estforce_step.pdf}
        \subcaption{Estimated Force:$\fest$}
        \label{fig4:crop-1115_diff_estforce_step}
    \end{minipage} 
    \begin{minipage}{\minipageratio\hsize}
    \centering
        \includegraphics[keepaspectratio, scale = \minifigscale]{contents/ForceControl/figure/1115/crop-1115_diff_force_step.pdf}
        \subcaption{Measured Force:$\fmsr$}
        \label{fig4:crop-1115_diff_force_step}
    \end{minipage}
    \caption{Step Response of Estimated Force and Measured Force}
    \label{fig4:crop-1115_diff_step}
\end{figure}
%%FIGURE
\begin{figure}[t]
    \begin{minipage}{\minipageratio\hsize}
    \centering
        \includegraphics[keepaspectratio, scale = \minifigscale]{contents/ForceControl/figure/1115/crop-1115_diff_estforce_sin.pdf}
        \subcaption{Estimated Force:$\fest$}
        \label{fig4:crop-1115_diff_estforce_sin}
    \end{minipage}
    \begin{minipage}{\minipageratio\hsize}
    \centering
        \includegraphics[keepaspectratio, scale = \minifigscale]{contents/ForceControl/figure/1115/crop-1115_diff_force_sin.pdf}
        \subcaption{Measured Force:$\fmsr$}
        \label{fig4:crop-1115_diff_force_sin}
    \end{minipage}
    \caption{Sine Responce of Estimated Force and Measured Force}
    \label{fig4:crop-1115_diff_sin}
\end{figure}
%%===========

次に,外乱を加えたときの応答について比較を行う.
外乱として\figname\ref{fig4:withBLWN-crop}に示すようにノイズをmatlab/simulink上で取得したhead側圧力$\phs$とrod側圧力$\prs$のセンサ値に付加する.
これにより,推定値$\fest$の値に外乱が乗ることになる.
用いるノイズは,matlab/simulinkのBand-Limited White Noiseを利用し,ノイズ強度0.01,サンプル時間0.001とした.

\begin{figure}[t]
    \centering
        \includegraphics[keepaspectratio, scale=1.0]{contents/ForceControl/figure/TikZ/withBLWN-crop.pdf}
        \caption{with Band-Limited White Noise}
        \label{fig4:withBLWN-crop}
\end{figure}

外乱を付加したときのステップ応答および正弦波応答についてそれぞれ\figname\ref{fig4:crop-1115_diffnoise_step}および\figname\ref{fig4:crop-1115_diffnoise_sin}に示す.
推定値および実測値のどちらにおいても,$H_\infty$制御の方がI-PD制御に比べて外乱による影響を抑えられていることから,\ref{sec:サーボ系H無限大制御器}で設計したコントローラは所望の性能を満たしているといえる.
特に,実測出力において外乱抑制がよく効いていることが確認できる.
波形の形から,推定値$\fest$から実測出力$\fmsr$までにはローパスフィルタの役割を果たす伝達関数が存在していることが予想でき,本研究におけるシステム同定で反映できなかった周波数帯にそれが存在すると考えられる.
\figname\ref{fig4:crop-1115_diff_force_sin}における目標値に対する遅れの要因もこの伝達関数によるものだと予想される.

さらに,外乱抑制は実測値の方によく効いていること,むだ時間を考慮した場合の方が若干ではあるが外乱をより抑えられていることが確認できる.
むだ時間のPad\'e近似の次数を大きくすることにより外乱抑制の効果も大きくなるが,その分コントローラの次数も上がることを考慮する必要があり,今回の場合では1次のPad\'e近似またはむだ時間を無視したコントローラでも抑制の効果は十分であると言える.
\begin{figure}[t]
    \begin{minipage}{\minipageratio\hsize}
    \centering
        \includegraphics[keepaspectratio, scale = \minifigscale]{contents/ForceControl/figure/1115/crop-1115_diffnoise_estforce_step.pdf}
        \subcaption{Estimated Force:$\fest$}
        \label{fig4:crop-1115_diffnoise_estforce_step}
    \end{minipage} 
    \begin{minipage}{\minipageratio\hsize}
    \centering
        \includegraphics[keepaspectratio, scale = \minifigscale]{contents/ForceControl/figure/1115/crop-1115_diffnoise_force_step.pdf}
        \subcaption{Measured Force:$\fmsr$}
        \label{fig4:crop-1115_diffnoise_force_step}
    \end{minipage}
    \caption{Step Responce of Estimated Force and Measured Force (with Band-Limited White Noise)}
    \label{fig4:crop-1115_diffnoise_step}
\end{figure}

\begin{figure}[t]
    \begin{minipage}{\minipageratio\hsize}
    \centering
        \includegraphics[keepaspectratio, scale = \minifigscale]{contents/ForceControl/figure/1115/crop-1115_diffnoise_estforce_sin.pdf}
        \subcaption{Estimated Force:$\fest$}
        \label{fig4:crop-1115_diffnoise_estforce_sin}
    \end{minipage} 
    \begin{minipage}{\minipageratio\hsize}
    \centering
        \includegraphics[keepaspectratio, scale = \minifigscale]{contents/ForceControl/figure/1115/crop-1115_diffnoise_force_sin.pdf}
        \subcaption{Measured Force:$\fmsr$}
        \label{fig4:crop-1115_diffnoise_force_sin}
    \end{minipage}
    \caption{Sine Responce of Estimated Force and Measured Force (with Band-Limited White Noise)}
    \label{fig4:crop-1115_diffnoise_sin}
\end{figure}